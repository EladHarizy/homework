\documentclass[fleqn]{article}
\usepackage[margin=1.5cm]{geometry}   % shrink margins
\usepackage{amsmath}    % math equation environments
\usepackage{amssymb}    % math symbols such as natural numbers N.
% \usepackage{tikz}	% for diagrams
% \usepackage{adjustbox}	% align enumerations containing tall objects to top. Usage: \item\adjustbox{valign=t}{...}
% \usepackage{centernot}	% centers not symbol. Usage: \centernot{...}

% Math mode in tables. Usage: use column type C
% \usepackage{array}   % for \newcolumntype macro
% \newcolumntype{C}{>{$}c<{$}} % math-mode version of "c" column type

% paragraph indentation within enumerations
\usepackage{enumitem}
\setlist{parsep=4pt,listparindent=\parindent}

% Configurations for logic proofs
\usepackage{logicproof, etoolbox}
\patchcmd{\logicproof}{\center}{\flushleft}{}{}
\patchcmd{\endlogicproof}{\endcenter}{\endflushleft}{}{}

\title{Mathematical Logic \\
\large Homework 9}
\author{Abraham Murciano}

\begin{document}

\maketitle

\begin{enumerate}

	\item % 1
	Define the following three propositions.
	\begin{gather*}
		\varphi_1 = \forall x \exists y (R(x, y) \lor R(y, x)) \\
		\varphi_2 = \forall x (\exists y R(x, y) \lor \exists y R(y, x)) \\
		\varphi_3 = (\forall x \exists y R(x, y)) \lor (\forall x \exists y R(y, x))
	\end{gather*}

	\begin{enumerate}
		\item % a
		\begin{enumerate}
			\item % i
			First we will show that \(\varphi_1 \to \varphi_2\).
			\begin{logicproof}{2}
				\forall x \exists y (R(x, y) \lor R(y, x)) & premise \label{100} \\
				\exists y (R(a, y) \lor R(y, a)) & \ref{100}, universal elimination \label{108} \\
				\begin{subproof}
					R(a, b) \lor R(b, a) & assumption \label{101} \\
					\begin{subproof}
						R(a, b) & assumption \label{102} \\
						\exists y R(a, y) & \ref{102}, existential introduction \label{104} \\
						\exists y R(a, y) \lor \exists y R(y, a) & \ref{104}, disjunction introduction \label{106}
					\end{subproof}
					\begin{subproof}
						R(b, a) & assumption \label{103} \\
						\exists y R(y, a) & \ref{103}, existential introduction \label{105} \\
						\exists y R(a, y) \lor \exists y R(y, a) & \ref{105}, disjunction introduction \label{107}
					\end{subproof}
					\exists y R(a, y) \lor \exists y R(y, a) & \ref{101}, \ref{102}--\ref{106}, \ref{103}--\ref{107} disjunction elimination \label{109}
				\end{subproof}
				\exists y R(a, y) \lor \exists y R(y, a) & \ref{108}, \ref{101}--\ref{109}, existential elimination \label{110} \\
				\forall x (\exists y R(x, y) \lor \exists y R(y, x)) & \ref{110}, universal introduction \label{111}
			\end{logicproof}
			Therefore by the deduction theorem we have
			\[\forall x \exists y (R(x, y) \lor R(y, x)) \to \forall x (\exists y R(x, y) \lor \exists y R(y, x))\]
			
			\item % ii
			Now we must show that \(\varphi_2 \to \varphi_1\).
			\begin{logicproof}{2}
				\forall x (\exists y R(x, y) \lor \exists y R(y, x)) & premise \label{200} \\
				\exists y R(a, y) \lor \exists y R(y, a) & \ref{200}, universal elimination \label{209} \\
				\begin{subproof}
					\exists y R(a, y) & assumption \label{201} \\
					\begin{subproof}
						R(a, b) & assumption \label{203} \\
						R(a, b) \lor R(b, a) & \ref{203}, disjunction introduction \label{204} \\
						\exists y (R(a, y) \lor R(y, a)) & \ref{204}, existential introduction \label{205}
					\end{subproof}
					\exists y (R(a, y) \lor R(y, a)) & \ref{201}, \ref{203}--\ref{205}, existential elimination \label{210}
				\end{subproof}
				\begin{subproof}
					\exists y R(y, a) & assumption \label{202} \\
					\begin{subproof}
						R(b, a) & assumption \label{206} \\
						R(a, b) \lor R(b, a) & \ref{206}, disjunction introduction \label{207} \\
						\exists y (R(a, y) \lor R(y, a)) & \ref{207}, existential introduction \label{208}
					\end{subproof}
					\exists y (R(a, y) \lor R(y, a)) & \ref{202}, \ref{206}--\ref{208}, existential elimination \label{211}
				\end{subproof}
				\exists y (R(a, y) \lor R(y, a)) & \ref{209}, \ref{201}--\ref{210}, \ref{202}--\ref{211} disjunction elimination \label{212} \\
				\forall x \exists y (R(x, y) \lor R(y, x)) & \ref{212}, universal introduction \label{213}
			\end{logicproof}
			Therefore by the deduction theorem we have
			\[\forall x (\exists y R(x, y) \lor \exists y R(y, x)) \to \forall x \exists y (R(x, y) \lor R(y, x))\]
		\end{enumerate}
		Since \(\varphi_1 \to \varphi_2\) and  \(\varphi_2 \to \varphi_1\), therefore \(\varphi_1\) and \(\varphi_2\) are equivalent.
		
		\item % b
		\begin{enumerate}
			\item % i
			Now we will demonstrate that \(\varphi_2 \not\to \varphi_3\) by stating a counterexample model. Let the universe for \(x\) and \(y\) be the real numbers in the interval \([0, 1]\), and let the predicate \(R(x, y)\) be \(x < y\).
			
			In this example, \(\varphi_2\) means that ``for all numbers in \([0, 1]\), there is a number in \([0, 1]\) which is smaller than it, or there is a number in \([0, 1]\) which is greater than it.'' In this case, \(\varphi_2\) is true, because 0 is smaller than every number in the interval except for 0 itself, and every number in the interval is greater than 0, except for 0 itself.
			
			However, \(\varphi_3\) means that ``either every number in \([0, 1]\) has a number greater than it, or every number in \([0, 1]\) has a number smaller than it.'' However this is not true, because there is no number in \([0, 1]\) which is greater than 1, so the first part of the proposition is false. The second part is also false, because there is no number in \([0, 1]\) which is less than 0. Therefore \(\varphi_2 \not\to \varphi_3\).
			
			\item % ii
			Now we will show that \(\varphi_3 \to \varphi_2\).
			\begin{logicproof}{1}
				(\forall x \exists y R(x, y)) \lor (\forall x \exists y R(y, x)) & premise \label{300} \\
				\begin{subproof}
					\forall x \exists y R(x, y) & assumption \label{301} \\
					\exists y R(a, y) & \ref{301}, universal elimination \label{302} \\
					\exists y R(a, y) \lor \exists y R(y, a) & \ref{302}, disjunction introduction \label{303}
				\end{subproof}
				\begin{subproof}
					\forall x \exists y R(y, x) & assumption \label{304} \\
					\exists y R(y, x) & \ref{304}, universal elimination \label{305} \\
					\exists y R(a, y) \lor \exists y R(y, a) & \ref{305}, disjunction introduction \label{306}
				\end{subproof}
				\exists y R(a, y) \lor \exists y R(y, a) & \ref{300}, \ref{301}--\ref{303}, \ref{304}--\ref{306} disjunction elimination \label{306} \\
				\forall x (\exists y R(x, y) \lor \exists y R(y, x)) & \ref{306}, universal introduction \label{307}
			\end{logicproof}
			Therefore by the deduction theorem we have
			\[(\forall x \exists y R(x, y)) \lor (\forall x \exists y R(y, x)) \to \forall x (\exists y R(x, y) \lor \exists y R(y, x))\]
		\end{enumerate}

		\item % c
		\begin{enumerate}
			\item % i
			Since \(\varphi_1 \leftrightarrow \varphi_2\) and \(\varphi_2 \not\to \varphi_3\), therefore \(\varphi_1 \not\to \varphi_3\).
			\item % ii
			Since \(\varphi_3 \to \varphi_2\) and \(\varphi_2 \to \varphi_1\), therefore \(\varphi_3 \to \varphi_1\).
		\end{enumerate}
	\end{enumerate}

	\item % 2
	\begin{enumerate}
		\item[(b)]
		\begin{gather*}
			\exists x (P(x) \land Q(x)) \to \forall x (P(x) \to Q(x)) \\
			= \forall x ((P(x) \land Q(x)) \to \forall y (P(y) \to Q(y))) \\
			= \forall x \forall y ((P(x) \land Q(x)) \to (P(y) \to Q(y)))
		\end{gather*}
	\end{enumerate}

	\item % 3
	Given the following propositions
	\begin{gather*}
		\alpha = \exists x (P(x) \to Q(x)) \\
		\beta = \forall x P(x) \to \exists x Q(x)
	\end{gather*}
	\begin{enumerate}
		\item[(b)]
		It is true that \(\beta \to \alpha\). A proof goes as follows.
		\begin{logicproof}{2}
			\forall x P(x) \to \exists x Q(x) & premise \label{400} \\
			\lnot \forall x P(x) \lor \exists x Q(x) & \ref{400}, material implication \label{401} \\
			\begin{subproof}
				\lnot \forall x P(x) & assumption \label{402} \\
				\exists x \lnot P(x) & \ref{402}, universal negation \label{403} \\
				\begin{subproof}
					\lnot P(a) & assumption \label{404} \\
					\lnot P(a) \lor Q(a) & \ref{404}, disjunction introduction \label{405} \\
					P(a) \to Q(a) & \ref{405}, material implication \label{406} \\
					\exists x (P(x) \to Q(x)) & \ref{406}, existential introduction \label{407}
				\end{subproof}
				\exists x (P(x) \to Q(x)) & \ref{403}, \ref{404}--\ref{407} existential elimination \label{408}
			\end{subproof}
			\begin{subproof}
				\exists x Q(x) & assumption \label{409} \\
				\begin{subproof}
					Q(a) & assumption \label{410} \\
					\lnot P(a) \to Q(a) & \ref{410}, disjunction introduction \label{411} \\
					P(a) \to Q(a) & \ref{411}, material implication \label{412} \\
					\exists x (P(x) \to Q(x)) & \ref{412}, existential introduction \label{413}
				\end{subproof}
				\exists x (P(x) \to Q(x)) & \ref{409}, existential elimination \label{414}
			\end{subproof}
			\exists x (P(x) \to Q(x)) & \ref{401}, \ref{402}--\ref{408}, \ref{409}--\ref{414} existential elimination \label{414}
		\end{logicproof}
	\end{enumerate}

	\item % 4
	\begin{enumerate}
		\item % a
		We must prove the following implication.
		\[\exists x P(x) \to \forall x Q(x) \vdash \forall x (P(x) \to Q(x))\]
		\begin{logicproof}{1}
			\exists x P(x) \to \forall x Q(x) & premise \label{500} \\
			\lnot \exists x P(x) \lor \forall x Q(x) & \ref{500}, material implication \label{501} \\
			\begin{subproof}
				\lnot \exists x P(x) & assumption \label{502} \\
				\forall x \lnot P(x) & \ref{502}, existential negation \label{503} \\
				\lnot P(a) & \ref{503}, universal elimination \label{504} \\
				\lnot P(a) \lor Q(a) & \ref{504}, disjunction introduction \label{505} \\
				P(a) \to Q(a) & \ref{505}, material implication \label{506} \\
				\forall x (P(x) \to Q(x)) & \ref{506}, universal introduction \label{507}
			\end{subproof}
			\begin{subproof}
				\forall x Q(x) & assumption \label{508} \\
				Q(a) & \ref{508}, universal elimination \label{509} \\
				\lnot P(a) \lor Q(a) & \ref{509}, disjunction introduction \label{510} \\
				P(a) \to Q(a) & \ref{510}, material implication \label{511} \\
				\forall x (P(x) \to Q(x)) & \ref{511}, universal introduction \label{512}
			\end{subproof}
			\forall x (P(x) \to Q(x)) & \ref{501}, \ref{502}--\ref{507}, \ref{508}--\ref{512} disjunction elimination QED
		\end{logicproof}
		
		\item % b
		We must prove that the following implication does not hold. We will do so by demonstrating a counterexample model.
		\[\forall x (P(x) \to Q(x)) \vdash \exists x P(x) \to \forall x Q(x)\]
		Let the universe for \(x\) be \(\mathbb{Z}\). Let the proposition \(P(x)\) mean that ``\(x\) is a multiple of 4'', and the proposition \(Q(x)\) mean that ``\(x\) is even''.

		It is true that for every number, if said number is a multiple of four, then it must be even. However it is untrue that if there exists a multiple of four --- which there does --- then all integers must be even.

		Therefore \(\forall x (P(x) \to Q(x)) \nvdash \exists x P(x) \to \forall x Q(x)\).
	\end{enumerate}

\end{enumerate}
    
\end{document}
