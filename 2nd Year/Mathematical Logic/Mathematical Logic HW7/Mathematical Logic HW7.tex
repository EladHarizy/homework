\documentclass[fleqn]{article}
\usepackage[margin=1.5cm]{geometry}   % shrink margins
\usepackage{amsmath}    % math equation environments
% \usepackage{amssymb}    % math symbols such as natural numbers N.
% \usepackage{tikz}	% for diagrams
% \usepackage{adjustbox}	% align enumerations containing tall objects to top. Usage: \item\adjustbox{valign=t}{...}
% \usepackage{centernot}	% centers not symbol. Usage: \centernot{...}

% Math mode in tables. Usage: use column type C
\usepackage{array}   % for \newcolumntype macro
\newcolumntype{C}{>{$}c<{$}} % math-mode version of "c" column type

% paragraph indentation within enumerations
\usepackage{enumitem}
\setlist{parsep=4pt,listparindent=\parindent}

% Configurations for logic proofs
% \usepackage{logicproof, etoolbox}
% \patchcmd{\logicproof}{\center}{\flushleft}{}{}
% \patchcmd{\endlogicproof}{\endcenter}{\endflushleft}{}{}

\title{Mathematical Logic HW7}
\author{Abraham Murciano}

\begin{document}
\maketitle

\begin{enumerate}

	\item % 1
	Let \(L\), \(S\), and \(P\) be the following predicates.
	\begin{gather*}
		L(x) = x \text{ is a lecturer at JCT.} \\
		S(x) = x \text{ is a student at JCT.} \\
		P(x, y) = x \text{ teaches \(y\).}
	\end{gather*}
	\begin{enumerate}
		\item[(b)]
		\(\exists a (L(a) \land \forall b (S(a) \to P(a, b)))\)

		\item[(d)]
		\(\exists a (S(a) \land \forall b (L(b) \to P(b, a)))\)

		\item[(f)]
		\(\exists a (L(a) \land \forall b (S(b) \to \lnot P(a, b)))\)
	\end{enumerate}

	\item % 2
	\begin{enumerate}
		\item[(b)]
		The universe is all the students and courses of Tal College.
		\begin{gather*}
			S(x) = x \text{ is a student.} \\
			F(x) = x \text{ is female.} \\
			C(x) = x \text{ is a course.} \\
			P(x, y) = x \text{ is in } y
		\end{gather*}
		\[\forall a (C(a) \to \forall b ((S(b) \land P(b, a)) \to F(b)))\]

		\item[(d)]
		The universe everything.
		\begin{gather*}
			P(x) = x \text{ is a person.} \\
			K(x, y) = x \text{ knows } y.
		\end{gather*}
		\[\exists a (P(a) \land \forall b (P(b) \to K(a, b))) \land \exists a (P(a) \land \lnot \forall b (P(b) \to K(b, a)))\]
	\end{enumerate}

	\item % 3
	\begin{enumerate}
		\item [(a)] 
		The following proposition
		\[\forall x (M(x) \to \exists y \exists z (H(y) \land H(z) \land F(x, y) \land F(x, z)))\]
		is equivalent to saying ``Every old person has two young relatives.''

		\item[(b)]
		The proposition ``One is young if and only if one is not old.'' can be formalized as follows.
		\[\forall a (H(a) \leftrightarrow \lnot M(a))\]

		\item [(d)]
		The proposition ``There is a person who is not young and not old, only if every person has relatives.'' can be formalized as follows.
		\[\exists a (\lnot H(a) \land \lnot M(a)) \to \forall a \exists b (F(a, b))\]
	\end{enumerate}

	\item % 4
	\begin{enumerate}
		\item[(b)]
		\(\exists a (S(a) \land \lnot \forall b ((T(b) \land T(a, b)) \to R(a, b)))\)

		\item[(d)]
		\(\forall a (S(a) \to (D(a) \to \forall b ((T(b) \land T(a, b)) \to R(a, b))))\)
	\end{enumerate}
\end{enumerate}
    
\end{document}
