\documentclass[fleqn]{article}
\usepackage{preamble}

\title{
	Statistics \\
	\medskip
	\large Homework 1 -- Representation of Data
}
\author{Abraham Murciano}

\begin{document}

\maketitle

\begin{answers}

	\item[1.]
	\begin{enumerate}
		\item % a
		For the following data set,
		\[1,8,1,5,8,6,3,3,3,7\]
		the mid-range, average, median and mode are as follows.
		\begin{align*}
			\text{Mid-range} &= \frac{1+8}{2} = 4.5 \\
			\text{Average} &= \frac{1+8+1+5+8+6+3+3+3+7}{10} = 4.5 \\
			\text{Median} &= \frac{3+5}{2} = 4 \\
			\text{Mode} &= 3
		\end{align*}

		\item % b
		For the following data set,
		\[14,18,30,31,15,18,27\]
		the mid-range, average, median and mode are as follows.
		\begin{align*}
			\text{Mid-range} &= \frac{14+31}{2} = 22.5 \\
			\text{Average} &= \frac{14+18+30+31+15+18+27}{7} \approx 21.86 \\
			\text{Median} &= 18 \\
			\text{Mode} &= 18
		\end{align*}
	\end{enumerate}

	\item[3.]
	In the month of August 2010 it was very hot. The Israeli meteorological service recorded the maximum temperature. The results are presented in Table \ref{q3-temperature}.
	\begin{table}
		\centering
		\begin{tabular}{||c|c||}
			\hline
			Number of days & Max. temperature (\degree C) \\
			\hline
			2	& 27-29 \\
			6	& 30-32 \\
			12	& 33-35 \\
			5	& 36-38 \\
			4	& 39-41 \\
			\hline
		\end{tabular}
		\caption{Maximum daily temperature frequencies for August 2010}
		\label{q3-temperature}
	\end{table}
	\begin{enumerate}
		\item % a
		Table \ref{q3-frequencies} shows the frequency, relative frequency, cumulative frequency, and relative cumulative frequency.
		\begin{table}[ht]
			\centering
			\begin{tabular}{||c||c|c|c|c||}
				\hline
				Max. temp. (\degree C) & Frequency & Relative frq. & Cumulative frq. & relative cumulative frq. \\
				\hline
				27-29 & 2	& 0.0690 & 2	& 0.0690 \\
				30-32 & 6	& 0.2069 & 8	& 0.2759 \\
				33-35 & 12	& 0.4138 & 20	& 0.6897 \\
				36-38 & 5	& 0.1724 & 25	& 0.8621 \\
				39-41 & 4	& 0.1379 & 29	& 1.0000 \\
				\hline
			\end{tabular}
			\caption{Many types of frequencies of the data in Table \ref{q3-temperature}}
			\label{q3-frequencies}
		\end{table}

		\item % b
		Figure \ref{q3-bar-graphs} shows a bar graph of the data in Table \ref{q3-temperature} as well as a histogram.
		\begin{figure}[h]
			\centering
			\begin{tikzpicture}
				\begin{axis}[
					bar chart,
					width = 6cm,
					bar width = 0.6cm,
					xlabel = Maximum temperature,
					ylabel = Frequency,
					symbolic x coords={27--29,30--32,33--35,36--38,39--41},
					x tick label style={rotate=45,anchor=north east}
					]
					\addplot coordinates {(27--29,2) (30--32,6) (33--35,12) (36--38,5) (39--41,4)};
				\end{axis}
			\end{tikzpicture}
			\begin{tikzpicture}
				\begin{axis}[
					histogram,
					width = 6cm,
					xlabel = Maximum temperature,
					ylabel = Frequency density,
					x tick label style={rotate=45,anchor=north east}
				]
					\addplot+[ybar interval] coordinates {(26.5,0.67) (29.5,2) (32.5,4) (35.5,1.67) (38.5,1.33) (41.5,0)};
				\end{axis}
			\end{tikzpicture}
			\caption{Bar graph of the data in Table \ref{q3-temperature} (left) and a histogram of the same (right)}
			\label{q3-bar-graphs}
		\end{figure}

		\item % c
		The mid-range, average, median and mode for this data are as follows.
		\begin{align*}
			\text{Mid-range} &= \frac{28 + 40}{2} = 34 \\
			\text{Average} &= \frac{2 \times 28 + 6 \times 31 + 12 \times 34 + 5 \times 37 + 4 \times 40}{29} \approx 34.3103 \\
			\text{Median} &= \frac{3 \left( \frac{29}{2} - 8 \right)}{12} + 32.5 = 34.125 \\
			\text{Modal class} &= [32.5,35.5)
		\end{align*}
	\end{enumerate}

	\item[6.]


\end{answers}

\end{document}
