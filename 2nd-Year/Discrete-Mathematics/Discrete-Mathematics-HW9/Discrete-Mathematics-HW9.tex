\documentclass[fleqn]{article}
\usepackage[margin=3cm]{geometry}   % shrink margins
\usepackage[margin=3cm]{caption}   % shrink captions
\usepackage{amsmath}    % math equation environments
\usepackage{amssymb}    % math symbols such as natural numbers N.

\newenvironment{answers}{ % same as enumerate but with more space between each answer
	\begin{enumerate}
		\setlength{\itemsep}{\bigskipamount}
}{\end{enumerate}}

\newcommand\Item[1][]{ % custom \Item command for block math
  \ifx\relax#1\relax  \item \else \item[#1] \fi
  \abovedisplayskip=0pt\abovedisplayshortskip=0pt~\vspace*{-\baselineskip}}

\usepackage{multicol}	% can be used to put enumerate in columns. Usage: \begin{multicols}{NumCols}\begin{enumerate}...

\newcommand*{\perm}[2]{{}^{#1}\!P_{#2}}
\newcommand*{\comb}[2]{{}^{#1}C_{#2}}

% \usepackage{tikz}	% for diagrams
% \usetikzlibrary{positioning}
% \usetikzlibrary{arrows.meta}

% \usepackage{adjustbox}	% align enumerations containing tall objects to top. Usage: \item\adjustbox{valign=t}{...}

% \usepackage{centernot}	% centers not symbol. Usage: \centernot{...}

% Math mode in tables. Usage: use column type C
% \usepackage{array}   % for \newcolumntype macro
% \newcolumntype{C}{>{$}c<{$}} % math-mode version of "c" column type

% paragraph indentation within enumerations
\usepackage{enumitem}
\setlist{parsep=4pt,listparindent=\parindent}

\title{Discrete Mathematics \\
\medskip
\large Homework 9}
\author{Abraham Murciano}

\begin{document}

\maketitle

\begin{answers}

    \item[1.]
    \begin{enumerate}
		\item[(d)]
		The number of integers in the interval between 1 and 1000 which are divisible by 6, 8, or 9 is equal to the size of the set of integers in the range divisible by 6, union the set of integers divisible by 8, union the set of integers divisible by 9.

		The size of the set of integers divisible by 6 (we shall call it \(A\)) is
		\[|A| = \left|\{6, 12, 18, \dots, 990, 996\}\right| = \left\lfloor\frac{1000}{6}\right\rfloor = 166\]

		The size of the set of integers divisible by 8 (we shall call it \(B\)) is
		\[|B| = \left|\{8, 16, 24, \dots, 992, 1000\}\right| = \left\lfloor\frac{1000}{8}\right\rfloor = 125\]

		The size of the set of integers divisible by 9 (we shall call it \(C\)) is
		\[|C| = \left|\{8, 18, 27, \dots, 990, 999\}\right| = \left\lfloor\frac{1000}{9}\right\rfloor = 111\]

		The size of the set of integers divisible by both 6 and 8 is equal to the size of the set of integers divisible by the lowest common multiple of these two numbers, which is 24.
		\[|A \cap B| = \left\lfloor\frac{1000}{24}\right\rfloor = 41\]

		The size of the set of integers divisible by both 6 and 9 is equal to the size of the set of integers divisible by the lowest common multiple of these two numbers, which is 18.
		\[|A \cap C| = \left\lfloor\frac{1000}{18}\right\rfloor = 55\]

		The size of the set of integers divisible by both 8 and 9 is equal to the size of the set of integers divisible by the lowest common multiple of these two numbers, which is 72.
		\[|B \cap C| = \left\lfloor\frac{1000}{72}\right\rfloor = 13\]

		The size of the set of integers divisible by 6, 8, and 9 is equal to the size of the set of integers divisible by the lowest common multiple of these three, which is 72. The size of this set (\(A \cap B \cap C\)) is also 13.

		Hence by the inclusion-exclusion principle,
		\begin{align*}
			|A\cup B\cup C|&=|A|+|B|+|C|-|A\cap B|-|A\cap C|-|B\cap C|+|A\cap B\cap C| \\
			&= 166+125+111-41-55-13+13 \\
			&= 306
		\end{align*}

		\item[(e)]
		We seek the number of numbers in the set \(\{1, \dots, 500\}\) which are divisible by 3, 5, 7, or 8. Let \(D_{a,b,\dots} = D_a \cup D_b \cup \dots\) be the set of numbers in the set \(\{1, \dots, 500\}\) which are divisible by all of \(a, b, \dots\)
		\begin{align*}
			|D_3| &= \left\lfloor\frac{500}{3}\right\rfloor = 166, &
			|D_5| &= \left\lfloor\frac{500}{5}\right\rfloor = 100, &
			|D_7| &= \left\lfloor\frac{500}{7}\right\rfloor = 71, \\
			|D_8| &= \left\lfloor\frac{500}{8}\right\rfloor = 62, &
			|D_{3,5}| &= \left\lfloor\frac{500}{15}\right\rfloor = 33, &
			|D_{3,7}| &= \left\lfloor\frac{500}{21}\right\rfloor = 23, \\
			|D_{3,8}| &= \left\lfloor\frac{500}{24}\right\rfloor = 20, &
			|D_{5,7}| &= \left\lfloor\frac{500}{35}\right\rfloor = 14, &
			|D_{5,8}| &= \left\lfloor\frac{500}{40}\right\rfloor = 12, \\
			|D_{7,8}| &= \left\lfloor\frac{500}{56}\right\rfloor = 8, &
			|D_{3,5,7}| &= \left\lfloor\frac{500}{105}\right\rfloor = 4, &
			|D_{3,5,8}| &= \left\lfloor\frac{500}{120}\right\rfloor = 4, \\
			|D_{3,7,8}| &= \left\lfloor\frac{500}{168}\right\rfloor = 2, &
			|D_{5,7,8}| &= \left\lfloor\frac{500}{280}\right\rfloor = 1, &
			|D_{3,5,7,8}| &= \left\lfloor\frac{500}{840}\right\rfloor = 0.
		\end{align*}

		Now we can obtain our solution.
		\begin{align*}
			D_3 \cup D_5 \cup D_7 \cup D_8 =\
			&|D_3| + |D_5| + |D_7| + |D_8| \\&
			- |D_{3,5}| - |D_{3,7}| - |D_{3,8}|
			- |D_{5,7}| - |D_{5,8}|
			- |D_{7,8}| \\&
			+ |D_{3,5,7}|
			+ |D_{3,5,8}|
			+ |D_{3,7,8}|
			+ |D_{5,7,8}| \\&
			- |D_{3,5,7,8}| \\
			=\ & 166 + 100 + 71 + 62 - 33 - 23 - 20 - 14 - 12 - 8 \\&
			+ 4 + 4 + 2 + 1 - 0 \\
			=\ & 300
		\end{align*}
	\end{enumerate}

	\item[2.]
	\begin{enumerate}
		\item[(b)]
		The number of ways in which twenty identical items can be distributed among five distinct compartments in such a way that no compartment contains exactly three items is equal to the total number of distributions minus the number of distributions in which at least one compartment contains precisely three items.

		The total number of distributions is
		\[\binom{20+5-1}{5-1} = \binom{24}{4} = 10626\]

		Now we must find the number of distributions in which at least one compartment contains precisely three items. Let \(D_x\) be the set of all possible distributions such that compartment \(x\) has precisely three items and let \(D_{a,b,\dots} = D_a \cap D_b \cap \dots\)

		We seek \(|D_1 \cup D_2 \cup D_3 \cup D_4 \cup D_5|\).

		Assuming \(a \neq b \neq c \neq d \neq e\),
		\begin{gather*}
			|D_a| = \binom{17+4-1}{4-1} = 1140 \\
			|D_{a,b}| = \binom{14+3-1}{3-1} = 120 \\
			|D_{a,b,c}| = \binom{11+2-1}{2-1} = 12 \\
			|D_{a,b,c,d}| = \binom{8+1-1}{1-1} = 1 \\
			|D_{a,b,c,d,e}| = 0 \\
		\end{gather*}

		Therefore the number of ways to distribute twenty identical items among five distinct compartments such that at least one compartment contains exactly three items is
		\begin{align*}
			|D_1 \cup D_2 \cup D_3 \cup D_4 \cup D_5| =\ &|D_1| + |D_2| + |D_3| + |D_4| + |D_5| \\
			&- |D_{1,2}| - |D_{1,3}| - |D_{1,4}| - |D_{1,5}| \\
			&- |D_{2,3}| - |D_{2,4}| - |D_{2,5}| \\
			&- |D_{3,4}| - |D_{3,5}| - |D_{4,5}| \\
			&+ |D_{1,2,3}| + |D_{1,2,4}| + |D_{1,2,5}| + |D_{1,3,4}| + |D_{1,3,5}| \\
			&+ |D_{1,4,5}| + |D_{2,3,4}| + |D_{2,3,5}| + |D_{2,4,5}| + |D_{3,4,5}| \\
			&- |D_{1,2,3,4}| - |D_{1,2,3,5}| - |D_{1,2,4,5}| - |D_{1,3,4,5}|  \\
			&- |D_{2,3,4,5}| + |D_{1,2,3,4,5}| \\
			=\ &5 \times 1140 - 10 \times 120 + 10 \times 12 - 5 \times 1 + 0 \\
			=\ &4615
		\end{align*}

		Therefore the number of ways to distribute twenty identical items among five distinct compartments such that at no compartment contains exactly three items is
		\[10626 - 4615 = 6011\]
	\end{enumerate}
\end{answers}

\end{document}
