\documentclass[fleqn]{article}
\usepackage{preamble}

\title{
	Statistics \\
	\medskip
	\large Homework 2 -- Measures of Dispersion
}
\author{Abraham Murciano}

\begin{document}

\maketitle

\begin{answers}

	\item[1.]
	\begin{enumerate}
		\item % a
		For the following set of observations for the random variable \(X\),
		\[\vec{x} = \{1,8,1,5,8,6,3,3,3,7\}\]
		the range is equal to
		\[\max(X) - \min(X) = 8 - 1 = 7\]

		The mean deviation is the average of all the deviations from the mean. (\(\xbar = 4.5\))
		\[\frac{\sum_{x \in \vec{x}}|x - \xbar|}{|\vec{x}|} = \frac{3.5+3.5+3.5+0.5+3.5+2.5+1.5+1.5+1.5+2.5}{10} = 2.4\]

		The sample variance of these observations is
		\begin{multline*}
			\frac{\sum_{x \in \vec{x}}(x - \xbar)^2}{|\vec{x}|} = \frac{3.5^2+3.5^2+3.5^2+0.5^2+3.5^2+2.5^2+1.5^2+1.5^2+1.5^2+2.5^2}{10} \\= \frac{12.25+12.25+12.25+0.25+12.25+6.25+2.25+2.25+2.25+6.25}{10} = 6.85
		\end{multline*}

		The standard deviation is simply the square root of the variance.
		\[s = \sqrt{6.85} \approx 2.62\]

		\item % b
		For the following set of observations for the random variable \(X\),
		\[\vec{x} = \{14,18,30,31,15,18,27\}\]
		the range is equal to
		\[\max(\vec{x}) - \min(\vec{x}) = 31 - 14 = 17\]

		The mean deviation is the average of all the deviations from the mean. (\(\xbar = \frac{153}{7} \approx 21.86\))
		\[\frac{\sum_{x \in \vec{x}}|x - \xbar|}{|\vec{x}|} = \frac{7.86+3.86+8.14+9.14+6.86+3.86+5.14}{7} = 6.41\]

		The sample variance of these observations is
		\begin{multline*}
			\frac{\sum_{x \in \vec{x}}(x - \xbar)^2}{|\vec{x}|} \approx \frac{7.86^2+3.86^2+8.14^2+9.14^2+6.86^2+3.86^2+5.14^2}{7} \\\approx \frac{61.78+14.90+66.26+83.54+47.06+14.90+26.42}{7} = 44.98
		\end{multline*}

		The standard deviation is simply the square root of the variance.
		\[s = \sqrt{44.98} \approx 6.71\]
	\end{enumerate}

	\item[4.]
	Given that \(a\) is the average of observations \(a_1, \dots, a_{18}\), and \(b\) is the average of \(a_1, \dots, a_9\), we are to find the average of \(a_{10}, \dots, a_{18}\) in terms of \(a\) and \(b\). Let \(c\) be this average.

	\[a = \frac{\sum_{i=1}^{18}a_i}{18} = \frac{\sum_{i=1}^{9}a_i + \sum_{i=10}^{18}a_i}{18} = \frac{\sum_{i=1}^{9}a_i}{18} + \frac{\sum_{i=10}^{18}a_i}{18}\]

	But we know that
	\begin{gather}
		b = \frac{\sum_{i=1}^{9}a_i}{9} = 2 \frac{\sum_{i=1}^{9}a_i}{18} \\
		c = \frac{\sum_{i=10}^{18}a_i}{9} = 2 \frac{\sum_{i=10}^{18}a_i}{18}
	\end{gather}

	Therefore we can now say that
	\begin{equation}
		a = \frac{b}{2} + \frac{c}{2} = \frac{b + c}{2} \label{q4}
	\end{equation}

	Now we can rearrange to obtain an expression for \(c\).
	\[c = 2a - b\]

	As a side-note, we have proven from equation (\ref{q4}) that the average of a set of observations is the average of the averages of each half.

	\item[6.]
	For the ordered observations \(x_1, \dots, x_n\), given that the absolute values of the standardised scores of \(x_1\) and of \(x_n\) are equal, the average \(\xbar\) of all the data is equal to the average of \(x_1\) and \(x_n\).
	\begin{equation}
		\xbar = \frac{x_1 + x_n}{2} \label{q6-x-bar}
	\end{equation}

	To prove this, let \(z_1\) and \(z_n\) be the standardised score for \(x_1\) and \(x_n\) respectively, and let \(\xbar\) be the average and \(s\) be the standard deviation.
	\begin{equation}
		z_1 = \frac{x_1 - \xbar}{s} \ \text{ and } \ z_n = \frac{x_n - \xbar}{s} \label{q6-z}
	\end{equation}

	Since \(x_1\) is the smallest value in our set of observations, \(x_1 \leq \xbar\). Conversely, \(x_n\) is the largest observation, so \(x_n \geq \xbar\). Additionally, \(s\) must always be positive. Therefore \(z_1 \leq 0\) and \(z_n \geq 0\). And since \(|z_1| = |z_n|\), we know that \(-z_1 = z_n\).

	Now we can rearrange the equations in (\ref{q6-z}) to obtain the following.
	\[
		x_1 = z_1 s + \xbar
		\ \text{ and } \
		x_n = z_n s + \xbar
	\]

	Now we can show that indeed equation (\ref{q6-x-bar}) is correct.
	\[\frac{x_1 + x_n}{2} = \frac{z_1 s + \xbar + z_n s + \xbar}{2} = \frac{2\xbar}{2} + \frac{s(z_1 + z_n)}{2} = \xbar + \frac{0}{2} = \xbar\]

	\item[7.]
	\begin{enumerate}
		\item % a
		We are to prove that
		\[s^2 = \frac{\sum_{i=1}^n {x_i}^2}{n} - \xbar^2\]

		Let us begin with what we know of \(s^2\).
		\[s^2 = \frac{\sum_{i=1}^n (x_i - \xbar)^2}{n}\]

		We can expand and rearrange this equation to obtain
		\begin{align*}
			s^2 & = \frac{\sum_{i=1}^n ({x_i}^2 - 2 x_i \xbar + \xbar^2)}{n}                                             \\
			    & = \frac{\sum_{i=1}^n {x_i}^2 - \sum_{i=1}^n 2 x_i \xbar + \sum_{i=1}^n \xbar^2}{n}                     \\
			    & = \frac{\sum_{i=1}^n {x_i}^2}{n} - \frac{\sum_{i=1}^n 2 x_i \xbar}{n} + \frac{\sum_{i=1}^n \xbar^2}{n} \\
			    & = \frac{\sum_{i=1}^n {x_i}^2}{n} - 2 \xbar \frac{\sum_{i=1}^n x_i}{n} + \frac{n \xbar^2}{n}            \\
			    & = \frac{\sum_{i=1}^n {x_i}^2}{n} - 2 \xbar \cdot \xbar + \xbar^2                                       \\
			    & = \frac{\sum_{i=1}^n {x_i}^2}{n} - 2 \xbar^2 + \xbar^2                                                 \\
			    & = \frac{\sum_{i=1}^n {x_i}^2}{n} - \xbar^2                                                             \\
		\end{align*}

		\item % b
		Given that for the following set of observations \(\{1, 5, 2, 3, m\}\), the variance \(s^2\) is 2, we seek the value of \(m\).

		First we will consider an equation for the variance.
		\begin{align*}
			s^2 = 2 & = \frac{\sum_{i=1}^n {x_i}^2}{n} - \xbar^2                                             \\
			        & = \frac{1^2 + 5^2 + 2^2 + 3^2 + m^2}{5} - \left( \frac{1 + 5 + 2 + 3 + m}{5} \right)^2 \\
			        & = \frac{39 + m^2}{5} - \left( \frac{11 + m}{5} \right)^2                               \\
			        & = \frac{195 + 5m^2}{25} - \frac{(11 + m)^2}{25}                                        \\
			        & = \frac{195 + 5m^2 - (11 + m)^2}{25}                                                   \\
		\end{align*}
		Now we rearrange this equation to find a pair of solutions for \(m\).
		\begin{align*}
			\frac{195 + 5m^2 - (11 + m)^2}{25}       & = 2  \\
			\Rightarrow 195 + 5m^2 - (11 + m)^2      & = 50 \\
			\Rightarrow 195 + 5m^2 - 121 - 22m - m^2 & = 50 \\
			\Rightarrow 4m^2 - 22m + 24              & = 0  \\
			\Rightarrow m \in \left\{ \frac{3}{2}, 4 \right\}
		\end{align*}

		\item % c
		If \(m = \frac{3}{2}\), then the median is equal to 2, otherwise, if \(m = 4\), the median is equal to 3.
	\end{enumerate}

	\item[8.]
	The data in Table \ref{q8-table} shows some statistics with respect to the starting salaries in	dollars of graduates of the University of Florida.

	\begin{table}[tb]
		\centering
		\begin{tabular}{||c|c||}
			\hline
			Percentile  & Salary   \\
			\hline
			7           & 17,000   \\
			18          & 20,000   \\
			25          & 21,000   \\
			50          & 26,000   \\
			75          & 30,375   \\
			\hline
			\hline
			Population  & 1,100    \\
			Sample size & 1,100    \\
			\hline
			Mean        & 26,064.2 \\
			\hline
		\end{tabular}
		\caption{Starting salaries of University of Florida graduates}
		\label{q8-table}
	\end{table}

	\begin{enumerate}
		\item % a
		The maximal salary among the 7\% of lowest paid workers is \$17,000.

		\item % b
		The interquartile range  is the \nth{75} percentile minus the \nth{25} percentile. So the \(\IQR = 30375 - 21000 = 9375\).

		\item % c
		The median is the value at the \nth{2} quartile, \(Q_2\), which is \$21,000.

		\item % d
		If every worker receives a 10\% increase, and then a \$300 bonus, the new average data would change to what is shown in Table \ref{q8d-table}.

		\begin{table}[b]
			\centering
			\begin{tabular}{||c|c||}
				\hline
				Percentile  & Salary    \\
				\hline
				7           & 19,000    \\
				18          & 22,300    \\
				25          & 23,400    \\
				50          & 28,900    \\
				75          & 33,712.5  \\
				\hline
				\hline
				Population  & 1,100     \\
				Sample size & 1,100     \\
				\hline
				Mean        & 29,355.62 \\
				\hline
			\end{tabular}
			\caption{Salaries after a 10\% increase and \$300 bonus}
			\label{q8d-table}
		\end{table}

		The new median would be the new \nth{50} percentile, which is
		\[\$21,000 \times 1.1 + \$300 = \$23,400\]

		The new mean is
		\begin{multline*}
			\xbar_2 = \frac{\sum_x xf(x)}{n} = \\
			\frac{19000 \times 77 + 22300 \times 121 + 23400 \times 77 + 28900 \times 275 + 33712.5 \times 275 + a_2 \times 275}{1100}
		\end{multline*}
		where \(a_2\) is the mark of the class between the \nth{75} percentile and the \nth{100} percentile after the raises have been given. To find \(a_2\) we must use the data given to us in Table \ref{q8-table}, where we were given the mean.
		\begin{multline*}
			\xbar_1 = 26064.2 = \\
			\frac{12000 \times 77 + 20000 \times 121 + 21000 \times 77 + 26000 \times 275 + 30375 \times 275 + a_1 \times 275}{1100}
		\end{multline*}

		Rearranging this equation gives us \(a_1 = 29841.8\), so
		\[a_2 = 29841.8 \times 1.1 + 300 = 33125.98\]

		Now we can calculate
		\[\xbar = \frac{23181537.5 + 275a_2}{1100} = \frac{23181537.5 + 275 \times 33125.98}{1100} = 29355.62\]
	\end{enumerate}


\end{answers}

\end{document}
